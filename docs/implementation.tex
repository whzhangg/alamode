\documentclass{article}
\usepackage{amssymb, amsmath, amsthm}
\usepackage[margin=1in]{geometry}
\usepackage{verbatim}
\usepackage{graphicx}
\usepackage{hyperref} % \url \href
\usepackage[]{algorithm2e}

\newtheorem{definition}{Definition}
\newtheorem{theorem}{Theorem}
\newcommand{\pfrac}[2]{\frac{\partial #1}{\partial #2}}

\begin{document}

\title{Solution to Linear Phonon BTE}
\author{Wenhao Zhang}
\date{\today}
\maketitle

\section{Formulation}
Using $q_1$ to denote phonon states $(q_1,n_1)$ and $\alpha$ as cartesian direction index, 
we write $\mathbf{b}$ as a vector with elements $b_{q,\alpha}$:
\begin{equation}
    b_{q_1,\alpha} = - \frac{1}{\beta} v_{q_1,\alpha} \pfrac{n_{q_1}^0}{T}
\end{equation}
We write the linear equation of phonon BTE with solution $f_{q_1}^i$ as:
\begin{align}
    b_{q_1,\alpha} = 
    &\left[ \sum_{q_2,q_3} \left( \Lambda_{q_1,q_2}^{q_3} + \frac{1}{2} \Lambda_{q_1}^{q_2, q_3} \right) + \frac{n_{q_1}^0(n_{q_1}^0+1)}{\tau_{q_1}} \right] 
    f_{q_1,\alpha} \notag \\
    &+ \sum_{q_2,q_3} 
    \left[ (f_{q_2,\alpha} - f_{q_3,\alpha})\Lambda_{q_1,q_2}^{q_3} - \frac{1}{2} (f_{q_2,\alpha} + f_{q_3,\alpha}) \Lambda_{q_1}^{q_2, q_3} \right]
\end{align}
where relaxation time $\tau_{q_1}$ is independent of the solution, and may include scattering from 
isotope defects, boundaries, or four phonon terms.

% TODO: write the matrix in better form

Now, let's rewrite this equation in a form suitable for iterative solution:
\begin{equation}
    \sum_{q_2} \left( P_{q_1,q_2} + M_{q_1,q_2} \right)  f_{q_2,\alpha} = b_{q_1,\alpha}
\end{equation} 
where $P_{q_1,q_2}$ and $M_{q_1,q_2}$ the elements of the $N_k \times N_k$ square matrix
$\mathbf{P}$ and $\mathbf{M}$. Especially, we call $\mathbf{P}$ the precondition matrix and it is given by:
\begin{align}
    \label{E:P}
    P_{q_1,q_2} &= D_{q_1,q_2} + \frac{n_{q_1}^0(n_{q_1}^0+1)}{\tau_{q_1}} \\
    D_{q_1,q_2} &= \delta_{q_1,q_2}\sum_{q_3,q_4} \left( \Lambda_{q_1,q_3}^{q_4} + \frac{1}{2} \Lambda_{q_1}^{q_3, q_4} \right)
\end{align}
And the matrix element of $\mathbf{M}$ given by:
\begin{align}
    &\sum_{q_2,q_3} 
    \left[ (f_{q_2,\alpha} - f_{q_3,\alpha})\Lambda_{q_1,q_2}^{q_3} - \frac{1}{2} (f_{q_2,\alpha} + f_{q_3,\alpha}) \Lambda_{q_1}^{q_2, q_3} \right] \notag \\
    &= \sum_{q_2,q_3} (\Lambda_{q_1,q_2}^{q_3} - \frac{1}{2}\Lambda_{q_1}^{q_2, q_3}) f_{q_2,\alpha}
    - \sum_{q_2,q_3} (\Lambda_{q_1,q_2}^{q_3} + \frac{1}{2}\Lambda_{q_1}^{q_2, q_3}) f_{q_3,\alpha} \notag \\
    &= \sum_{q_2,q_3} (\Lambda_{q_1,q_2}^{q_3} - \frac{1}{2}\Lambda_{q_1}^{q_2, q_3}) f_{q_2,\alpha}
    - \sum_{q_2,q_3} (\Lambda_{q_1,q_3}^{q_2} + \frac{1}{2}\Lambda_{q_1}^{q_3, q_2}) f_{q_2,\alpha} \notag \\
    &= \sum_{q_2,q_3} (\Lambda_{q_1,q_2}^{q_3} - \Lambda_{q_1,q_3}^{q_2} - \Lambda_{q_1}^{q_3, q_2}) f_{q_2,\alpha} \notag \\
    &= \sum_{q_2} M_{q_1,q_2} f_{q_2,\alpha} 
\end{align}
Giving 
\begin{equation}
    \label{E:M}
    M_{q_1,q_2} = \sum_{q_3} (\Lambda_{q_1,q_2}^{q_3} - \Lambda_{q_1,q_3}^{q_2} - \Lambda_{q_1}^{q_3, q_2})
\end{equation}
In the iteration process, we have (in matrix form):
\begin{gather}
    ( \mathbf{P} + \mathbf{M} )  \mathbf{f}_{\alpha} = \mathbf{b}_{\alpha} \\
    \mathbf{P} \mathbf{f}_{\alpha}^{i+1} = \mathbf{b}_{\alpha} - \mathbf{M} \mathbf{f}_{\alpha}^i
\end{gather} 
Defining the residual:
\begin{equation}
    \mathbf{r}_{\alpha}^i =  \mathbf{b}_{\alpha} - ( \mathbf{P} + \mathbf{M} )  \mathbf{f}_{\alpha}^i
\end{equation}
and its euclidean norm $\|\mathbf{r}^i\|$, solution of the linear equation can be found by the following algorithm
\ref{Alg1}

\begin{algorithm}[t]
    \caption{Iterative solution}
    \label{Alg1}
    \KwData{matrix $\mathbf{P}$, $\mathbf{M}$, vector $\mathbf{b}$, convergence threshold $\epsilon$, maximum step $N$}
    initialize $\mathbf{f}_{\alpha}^0 = 0$\;
    initialize residual $\mathbf{r}_{\alpha} = \mathbf{b}_{\alpha}$\;
    step i = 0\;
    \While{$\|\mathbf{r}\|/\|\mathbf{b}\| < \epsilon$ and $i < N$}{
        $\mathbf{f}_{\alpha}^{i+1} = \mathbf{f}_{\alpha}^{i} + \mathbf{P}^{-1} \mathbf{r}_{\alpha} $\;
        $\mathbf{r}_{\alpha} =  \mathbf{b}_{\alpha} - ( \mathbf{P} + \mathbf{M} )  \mathbf{f}_{\alpha}^i$\;
        $i = i + 1$\;
    }
\end{algorithm}

The related equation in non-matrix form is:
\begin{align}
    \|\mathbf{r}\| &= \sqrt{\sum_{q_1,\alpha} r_{q_1,\alpha}^2} \\
    f_{q_1,\alpha}^{i+1} &= f_{q_1,\alpha}^{i} +  r_{q_1,\alpha} / P_{q_1,q_1} \\
    r_{q_1,\alpha} &= b_{q_1,\alpha} - \sum_{q_2} \left( P_{q_1,q_2} + M_{q_1,q_2} \right)  f_{q_2,\alpha}^{i+1}
\end{align}
where the matrix elements are given by equation \eqref{E:P}, \eqref{E:M}

\section{Symmetry adapted}


\end{document}
